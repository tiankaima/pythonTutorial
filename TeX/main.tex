% !TeX encoding = UTF-8
% !TeX program = xelatex
% !TeX spellcheck = en_US

\documentclass{book}
\usepackage[UTF8]{ctex}
\usepackage{listings}
\usepackage{tcolorbox}
\tcbuselibrary{skins, breakable, theorems}
\usepackage{frame}
\lstset{
	basicstyle          =   \sffamily,
	keywordstyle        =   \bfseries,
	commentstyle        =   \rmfamily\itshape,
	stringstyle         =   \ttfamily,
	flexiblecolumns,
	numbers             =   left,
	showspaces          =   false,
	numberstyle         =   \zihao{-5}\ttfamily,
	showstringspaces    =   false,
	captionpos          =   t,
	frame               =   lrtb,
}

\lstdefinestyle{Python}{
	language        =   Python,
	basicstyle      =   \zihao{-5}\ttfamily,
	numberstyle     =   \zihao{-5}\ttfamily,
	keywordstyle    =   \color{blue},
	keywordstyle    =   [2] \color{teal},
	stringstyle     =   \color{magenta},
	commentstyle    =   \color{red}\ttfamily,
	breaklines      =   true,
    columns         =   fixed,
	basewidth       =   0.5em,
}
\title{\LARGE Python 入门教程\\\small Python for beginners}
\author{TianKai Ma}
\date{\today}
\setcounter{chapter}{-1}

\begin{document}
\maketitle
\tableofcontents
\chapter{在开始之前}
\setcounter{section}{-1}

\section{引言}
这个教程是一个尝试:以课程的形式,面向高三毕业生(或者其他感兴趣的人)讲述一些有关Python的入门知识。

你现在看到的这篇文档,就是这个教程的讲义。

有关开设这样一门课程:一方面,是想通过这样的方式锻炼一下自己的能力,顺带考验一下自己对于这样一门语言的熟练程度;另一方面,也是注意到了高三的毕业生确有这方面的需求,但同时市面上又很少有适合这些人的入门内容。(这句话的意思是:许多开发文档、以及各种帮助新手入门的文章,尽管优秀,都不太适合这些完全没有经验的读者,会走很多没必要的弯路,同时网络上的课程质量大多良莠不齐,很难推荐)

无论如何,我希望能通过这样的一次尝试,培养一些对 Coding 感兴趣的同学,让他们在日后的学习、工作上带着这些工具和灵感。

\section{课程目标}

Python 并不是一门很难掌握的语言。也正因如此,它才适合新入门编程语言的同学。从这个角度来说,什么叫做真正“入门 Python ”值得仔细讨论。

我对这门课程的目标是这样的:
\begin{itemize}
	\item 熟练掌握Python的使用和基本语法
	\item 掌握基础的调试项目的能力
	\item 掌握“抽象-建模-编程-解决”问题的能力
	\item 训练解决问题,掌握“提问的艺术”
	\item 理解开源精神
\end{itemize}

同时,考虑到受众,主要还是照顾没有任何基础的同学,在课程的前半部分会尽量放慢速度,细致地讲解一些基础内容。课程后半部分会更倾向于实践主导:提供一些简单的项目目标以供实践,并在开发受阻的时候给予适当帮助。

\section{课程安排}

\begin{tcolorbox}
	暂定时间:2022.7.11 - 2022.8.12 每周的周一和周五(具体时间待定)
\end{tcolorbox}

每次课安排2小时左右,共计约20课时。以课程群的形式组织,安排一名助教(我也不清楚作用在哪里,答疑课?),暂定以腾讯会议的方式授课,完成后会将录像处理后上传Bilibili平台。

不设任何考勤、考核内容。

\section{在第一次课程开始之前}

需要准备的东西有:
\begin{itemize}
	\item 充足的时间和耐力(不要半途而废啊kora)
	\item 一台正常工作的电脑(装有Windows 10+/macOS 10.15+)
	\item 钻研的精神,最好以此为动力
\end{itemize}

\textbf{还需要做的准备工作:安装PyCharm}
\begin{tcolorbox}
	$\Rightarrow $Instructions:

	打开 https://www.jetbrains.com/pycharm/download/,

	选择对应的版本下载、安装(注,无特殊需要,请选择Community版本)
\end{tcolorbox}
\end{document}